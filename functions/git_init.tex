\begin{frame}{Git init}
    \begin{block}{Git init, or, where do baby repositories come from}
    \textbf{In a nutshell}, you use git init to make an existing directory of content into a new Git repository.
    You can do this in any directory at any time, completely locally.
    \lstinputlisting{code/gitinit.sh}
    \end{block}
    \clfact{git init}{initializes a directory as a Git repository}
\end{frame}
\demo{Creating our repository}
